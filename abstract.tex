In this bachelor thesis, we deal with few-shot learning on image datasets. It is important to learn data with desirable accuracy and prevent overfitting while there are few samples obtainable. Data augmentation will help us to reach this goal.

We will introduce four techniques of data augmentation that generate synthetic data from few-shot
datasets to improve the accuracy of learning and prevent overfitting. The synthetic data will be generated
with the help of various transformations and deformations on sample images.

In this work, we will compare and investigate the performance of each technique of data augmentation
on different image datasets. Based on the investigation, we will introduce three new techniques of
data augmentation that have a better performance than the other ones.

\begin{bigquote}{John McCarthy}
    If it takes 200 years to achieve artificial intelligence and then finally there's a textbook that explains how it's done, the hardest part of that textbook to write will be the part that explains why people didn't think of it 200 years ago.
\end{bigquote}

